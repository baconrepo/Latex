
\documentclass[twocolumn,11pt]{article}
\setlength{\textheight}{9truein}
\setlength{\topmargin}{-0.9truein}
\setlength{\parindent}{0pt}
\setlength{\parskip}{10pt}
\setlength{\columnsep}{.4in}
\usepackage{array}
\newcommand{\beq}{\begin{equation}}
\newcommand{\eeq}{\end{equation}}
\renewcommand{\abstractname}{Band Gaps of Silicon and Unknown Semiconductor }
\usepackage{graphicx}
\usepackage{url}
\title{Thermal Properties of Silicon and an Unknown Semiconducting Diode}
\author{Blake Cole}
\date{4 March 2019}
\begin{document}
\pagestyle{plain}
 \twocolumn[
   \begin{@twocolumnfalse}
   \maketitle
   \setlength{\parindent}{0pt}
   \begin{abstract}The electrical properties of a semiconductor are dependent on the elemental components of the device and the temperature at which it operates. This expirement studied the conducting properties of a Silicon diode, and simultaneously the resistive properties of an unknown thermistor across a wide range of temperature. From this the ideality factor of the Silicon diode was determined to be 1.76, an indicator of how well it conducts current as function of voltage, compared to values of 1.5 to 2.5 for commercially availabe diodes. Additionally the unknown semiconducting metal is thought to be Germanium, after calculating the band gap to be 0.58 Electron Volts (eV). 
		\vspace{.3in} 
     \end{abstract}
    \end{@twocolumnfalse}]

\section*{Theory}
Many metals are known to be electrically conductive, such as copper and aluminium, commonly thought to be due to the materials ability to transfer electrons by covalent bonds between atoms. This property of metals and even some nonmetals is better explained by the energy band theory of solid state physics, based on the Pauli exclusion principle, which states that no two electrons can share the same quantum numbers, and due to the large numbers of atoms present in a solid. The theory predicts that the many electron states will form energy levels, since the electrons in a solid are at nearly the same energy these levels form stacks of energy called bands. At a certain height, the electrons are in a state which they are able to participate in conduction, the so called conduction band. In conductive metals there are many electrons near or at this level. In insulating materials the difference between the resting equilibrium energy level, the valence band, and the conducting band is very large, on the order of 10s of eV's. For example the air requires an enermous amount of energy to conduct electricity through it. As the name implies, a semiconductor is somewhere in between. Crucially the energy of these electrons can be predicted according to quantum mechanics, specifically the Shockley Equation which predicts that the current is proportional the energy levels of the electrons, described by Fermi-Dirac statistics. The Shockley Equation is as follows: 
\beq
I=I_0\left(e^{qV/K_BT}\right)
\label{eq:Shockley}
\eeq
where $q$ is the charge, $V$ is the voltage, $K_B$ is Boltzmann's constant, and $T$ is the temperature in Kelvin.

The Shockley Equation has been revised to correct for nonideal currents which include tunneling currents, surface currents, and recombination-generation currents. These effects can be summarized through the addition of the variable $\eta$ called the ideality factor, which represents the nearness of the diode to an ideal diode with a value of $\eta =1$. This is shown in Equation \ref{eq:DiodeEquation}, in what is known as the "Diode Equation"
\beq
I=I_0\left(e^{qV/\eta{}K_BT}\right)
\label{eq:DiodeEquation}
\eeq
where
\beq
I_0 \propto f\left(T\right)e^{-E_g/\eta{}K_BT}
\label{eq:Eq9}
\eeq
and where $E_g$ is the energy band gap.
Taking the natural log of Equation \ref{eq:DiodeEquation} shows that the relationship between the natural log of current and voltage will be linear with a slope $m=q/\eta{}K_BT$ and an intercept of $I_0$. Thus varying the temperature will result in different slopes, different proportions between current and voltage. If these slopes are then plotted against 1/Temperature, this line will have a slope of $m=q/\eta{}K_B$ and using the known values of $q$ and $K_B$ the quality factor can be extracted. Equation \ref{eq:Eq9} can then be used to find the energy band gap $E_g$ of the diode. Plotting $ln\left(I_0\right)$ versus 1/Temperature, has a slope $m=-E_g/\eta{}K_B$. Using Boltzmann's constant and the quality factor previously found, $E_g$ can be calculated from this slope.
Similarly the resistivity of a substance can be calculated using the related Boltzmann statistics and the following equation.
\beq
\rho{}=A\left(T\right)e^{E_g/2K_BT}
\label{eq:Resistivity}
\eeq
In the same fashion as before, the energy gap of the material can be found by plotting the natural log of resistance versus the temperature reciprocal. From the energy gap, the identity of the substance can be estimated by comparing to known values.
\section*{Experiment}
In order to observe the predicted temperature dependance, the semiconductors are encased in a copper block along with a temperature probe and submerged in various temperature baths. The solution and corresponding temperature is recorded below.


\begin{tabular}{|l|l|l|l|l|}
\hline
  Description & Celsius & Kelvin & Farhenhiet \\ \hline
 \hline
  Room Temperature & 20 & 293 & 68\\ \hline
  \hline
   Ice Water & 1 & 274 & 33.8 \\ \hline
  \hline
   Dry Ice and Isopropyl &-60 & 213 & -76 \\ \hline
  \hline
   Boiling Water & 100 & 373 & 212\\ \hline
  \hline
   Boiling Silicon Oil & 160 & 433 & 320 \\ \hline
  \hline
\end{tabular}


The quality factor $\eta$ can be extracted by creating IV curves at different temperatures. The circuit is wired using a 1N4001 silicon diode in forward bias as shown, along with a seperate connection from an ohmmeter to the unknown semiconductor. 
\begin{figure}[!h!t]
	\centering
		\includegraphics[width=3in]{circuit.png}
	\caption{Diode Circuit}
	\label{fig:circuit}
\end{figure} 

A Fluke digital multimeter was used to monitor resistance along with a Keithley 2000 precision multimeter to measure current in the diode. The instruments were digitally recorded during the expirement, utilizing the semiconductors in an iPhone in order to accurately track rapid changes.
\begin{figure}[!h!t]
	\centering
		\includegraphics[width=2.5in]{IMG_0527.jpg}
	\caption{Expiremental set up}
	\label{fig:setup}
\end{figure} 

Once the temperature is stable, the supplied current is varied and the resulting voltage is recorded. In this expirement 20 data samples were taken with a current ranging from 0-100 milliamps (mA). 
In this expirement there are relative safety hazards with the materials used. If water becomes trapped in the copper block and then heated in boiling oil, which can reach temperatures of $200^\circ$C, it will rapidly expand causing the oil to splash. Adequate insulation needs used when dealing with solid C02.
\section*{Analysis}
In order to extract the quality factor $\eta $ of the Silicon diode, the voltage and current at various temperatures is plotted on an xy-axis. According to Equation \ref{eq:DiodeEquation} the slope of these lines is the known relation $m=q/\eta{}K_BT$. The lines were fit and Figure \ref{fig:IV curves} was generated using gnuplot software.
\begin{figure}[!h!t]
	\centering
		\includegraphics[width=3in]{yvsxplot.png}
		\caption{IV Curves for Silicon Diode}
	\label{fig:IV curves}
\end{figure} 


Plotting the previously calculated slopes $m$ and intercepts versus 1/Temperature, shown in Figure \ref{fig:IVslopes}, and using Equation \ref{eq:Eq9}, the quality factor $\eta$ of Silicon can be calculated. It was found to be 1.76 for the 1N4001 silicon diode. Compared with commercially available diodes, which have a quality factor of between 1.5-2.5, this seems like an accurate measurement within the range of reasonable values.
\begin{figure}[!h!t]
	\centering
		\includegraphics[width=3in]{ivslopes.png}
		\caption{IV Slopes and Intercepts vs 1/T}
	\label{fig:IVslopes}
\end{figure} 







The resistance data collected from the unknown material can be used to generate a plot of resistance versus 1/Temperature, and from that extract the energy gap $E_g$ as previously described in Equation \ref{eq:Resistivity}, and again fit using gnuplot to find slope.




\begin{figure}[!h!t]
	\centering
		\includegraphics[width=3in]{rvstnew.png}
		\caption{Resistance (Ohms) vs 1/T}
	\label{fig:rsvt}
\end{figure} 
From this fit line, of which $m=E_g/2K_B$, the energy of the band gap $E_g$, can be calculated using the known value of Boltzmann's constant. In this expirement the slope was found to be 3361 which yields an energy gap of 0.58 eV. Comparing with known values and commonality, leads to the conclusion that the unknown semiconductor was made of Germanium.
The observed behaviour of the unknown semiconductor and the Silicon diode appear to be accurately predicted using the methods and theory previously described, which accurately explains the otherwise mysterious phenomenon of electrical conductivity increasing proportionally with temperature. 


\section*{References}

"Semiconductor." Wikipedia, The Free Encyclopedia. Wikipedia, The Free Encyclopedia, 19 Mar. 2019. 

\url{http://hyperphysics.phy-astr.gsu.edu/hbase/magnetic/magcur.html}
\end{document}
