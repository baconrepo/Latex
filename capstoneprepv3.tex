\documentclass[aps,prb,twocolumn,groupedaddress,nofootinbib,floatfix]{revtex4}
%
\usepackage{graphicx} 
\usepackage{graphics}
\usepackage{epsfig}
\newcommand{\beq}{\begin{equation}}
\newcommand{\eeq}{\end{equation}}
%\setlength{\skip\footins}{0.3in}
\begin{document}
%
\title{Improving Expiremental Efficacy of Studying Brownian Motion}

%
\author{Blake cole}
%
% DON'T CHANGE ANYTHING IN THE NEXT FEW LINES OR DELETE BLANK LINES
%
\affiliation{This work was submitted as part of a course requirement for completion of the BS degree in the Physics Program at RIT and, in its current form, does not appear in any publication external to RIT.}
%
% PUT YOUR ADVISOR NAME BELOW.  DON'T DELETE ANY LINES
%
\altaffiliation [Rochester Institute of Technology, School of Physics and Astronomy, Faculty Advisor: ]{Dr. Louis McLane}

\date{\today}

\begin{abstract} \noindent Brownian Motion was first observed by Robert Brown

The efficacy of Brownian Motion expirements has improved significantly since the advent of digital recording devices. In the past, this phenomenon was observed by viewing suspended particles through a microscope and camera lucida, and tracking the particles by marking paper at regular intervals. Utilizing high frame rate digital cameras and specialized tracking software improves the efficacy of this expirement significantly. So much so, that this method can now be used to study the visco-elastic properties of the solution the particles are suspended in. This project aims to create an expiremental process for studying Brownian Motion in an undergraduate laboratory setting, as well as potentially performing microrheology expirements.  

Place your abstract here.  It should describe your project, what you propose to do, and any specific questions you hope to address.   Your abstract should stand alone, and make sense without the rest of the paper.   Thus you should not refer to your figures or tables.   Abstracts are frequently published without the rest of the paper.   Your Capstone II final paper will have the abstract and author info, for example, available on the COS webpage - but not the rest of the paper.
\end{abstract}

\maketitle

\section*{Introduction}
Brownian Motion was first observed by Robert Brown when he saw chaotic motion of tiny pollen particles suspended in water, originally he thought this might be due to biological organisms. Albert Einstein later theorized that this motion resulted entirely from the atomic nature of matter and could be explained as collisions between the particle and the molecules in the solution. Einstein's hypothesis was tested by the French physicist, Jean Baptiste Perrin, who had developed a technique for creating precise spherical particles around $1 \times 10^{-6}$ meters (1$\mu$m) in size. To observe the chaotic motion, the particles were suspended in water and observed through a microscope and camera lucida, which allowed the image of the microscope slide to be superimposed on a drawing surface. The position of the particles were then marked at regular time intervals. This experimental process can be greatly improved by using a digital camera and specialized particle tracking software. 

The motion of the particle can be modeled as the sum of the impulses during a given instance ($dt$) on the particle. Consider the interaction between one molecule of the solution and the particle being studied, during the collision it imparts a force $F_i(t)$ to the particle. Over a given time $dt$ the impulse is given by:
\beq
J_i=\int F_i(t)dt
\label{eq:1}
\eeq
Given the relative size difference between the particle and the individual molecules which comprise the solution, one can imagine many thousands of impulses in a short time $dt$. 
The force on the particle is then given by the sum of all such impulses over a time $dt$.
\beq
F(t)dt=\sum_{i}J
\label{eq:2}
\eeq
Where $F(t)$ is given by Paul Langevin's equation describing a viscous fluid(REFER):
\beq
F(t)=-\alpha{}v(t)dt+F^{(r)}(t)dt
\label{eq:Langevin Force}
\eeq
Where $\alpha$ is the viscous drag coefficient, studied by Stokes and found to be:
\beq
\alpha{}=3\pi{}\eta{}d
\label{eq:drag coeff}
\eeq
for sphere of diameter $d$ in a fluid with viscosity given by $\eta$ (REF BM4). This term causes the particle's velocity to tend towards zero, while the $F^{(r)}(t)$ represents the random force applied by thousands of particulate collisions, and in turn causes it to accelerate once again, giving rise to the particle's chaotic motion. The randomly applied force vector $F^{(r)}(t)$ creates a distribution in the particle's velocity. Specifically, one that satisfies the equipartition theorem, which states that the kinetic energy of the molecules in a system is proportional to the temperature.  In one dimension this can be expressed as:
\beq
\frac{1}{2}m\langle{}v_{x}^{2}\rangle=\frac{1}{2}K_{B}T
\label{eq:equipartition}
\eeq
where $m$ is the mass, $T$ is the temperature, $K_B$ is the Boltzmann constant, and $\langle{}v_{x}^{2}\rangle$ is the average velocity for a large population size.
The governing statistics behind this motion come from an application of the central limit theorem, which states that if many numbers are randomly drawn from the same probability distribution, the sum of these numbers will be a Gaussian-distributed random number (ref).

Specifically the theorem predicts that $N$ random numbers drawn from any probability distribution, with a mean of $\mu_i$ and variance $\sigma_{i}^2$ then the sum of those numbers $\sum N$ will result in a Gaussian distribution with mean $\mu = N\mu_i$ and a variance of $\sigma^2=N\sigma_{i}^2$. 

In the study of Brownian Motion the mean displacement ($\Delta{}r$) of the particle is equal to zero, since the displacement in either direction is equally likely. However the mean squared displacement (MSD) is the variance of the resulting Gaussian. In one dimension:
\beq
\sigma^2=\langle{}\Delta{}r^2\rangle{}=\langle{}\Delta{}x^2\rangle{}
\label{eq:Displacement}
\eeq


Einstein was the first to theorize a correlation between the Diffusion Coefficient ($D$) and the variance of the distribution (CITE BM 11)



It is recommended that you start with an Introduction or Background
section.  When you cite things \cite{Foner1} put the citations in a separate file and
use bibtex.   It is good to have both historical and modern references. \cite{Tickle,Basso}
Sometimes you need to cite things other than articles. \cite{UnitsTable}
%
\section*{Data Collection and Analysis}
These predictions can be reproduced utilizing a digital camera with a fast frame rate and a microscope with sufficient zoom. In general the procedure will require the digital camera to be sufficiently mounted to the microscope lens in such a way that the particles can be recorded reliably. Next a microscope slide containing a sample of microsphere particles will be prepared in such a way to minimize wind currents. The data will be collected and then analyzed, likely using a version of specialed tracking software developed by Eric Weeks and others. (REFER). The resulting displacement data can be fit to a Gaussian and the variance calculated in order to extract the diffusion coefficient according to equations \ref{eq:Variance} and (7). Additionaly the two-dimensional MSD can be calculated at each time, relative to the camera frame rate, using the following equation:
\beq
\langle\Delta{}r^{2}(t)\rangle{}=N_{i}^{-1}\sum_{i=1}^{N_i}\left[\left(x\left(t_{i}+t\right)-x\left(t_i\right)\right)^2+\left(y\left(t_{i}+t\right)-y\left(t_i\right)\right)^2\right]
\label{eq:two2dMSD}
\eeq
A collective ensemble of the mean squared displacements is averaged and the slope of the line used to calculate the diffusion coefficient.

\section*{Project Goals}	
The main goal of this capstone project is to create and document an experimental procedure, so that the experiment can be performed by undergraduates in a physics laboratory class. As such, a complete documentation for this expiremental procedure will be created for reproducibility and use in the lab course. Portions of this project will be dedicated to error analysis and potential pitfalls that may be encountered, in order to create a reliable and repeatable experiment. Additionally, if the expirement proves reliable, a similar technique could be used to study the visco-elastic properties of unknown solutions, in a process known as microrheology. 

\section*{Budget}
This project includes a budget proposal of around ---, which covers some specialty items including a digital camera, microspheres, and a fluid to increase the viscosity of the solution. 

Those key components and other items needed to conduct the expirement are listed below:

 \begin{tabular}{|l|r|}
\hline
  Bausch and Lomb Microscope & Lent from department\\ \hline

  Digital Camera DMK 33UX178 & 40 \\ \hline
  
  Microsphere Solution & 200  \\ \hline
 
  Diluting/Viscous Agent & 30  \\ \hline
  \hline
   Boiling Water & 100 \\ \hline
  \hline
   Boiling Silicon Oil & 160 \\ \hline
  \hline
\end{tabular}





Combining the statistical techniques described above, with high quality video equipment, will allow a thorough analysis of Brownian Motion. In order to achieve the desired results the following procedure will be designed and implemented for use in an undergraduate laboratory class. 




You should have several sections in here; this is the body of your report.   Do not number the sections.
Use of the asterisk in the section heading eliminates numbering.   You probably won't need subsections.
%
\section*{Notes on Figures}
It is also possible/likely that you will have figures in this or other
sections.  \textbf{Figures must be your own or you
must have permission from the journal/author).}  Because provenance
cannot be determined from Wikipedia or other popular websites, use of
figures from these sites is prohibited.   

It is crucially important that figures have large enough type font that you can read the axis labels and any insets.   Test this before preparing many figures.   It is surprisingly difficult to make the type big enough to still be readable once reduced to two-column format size.

Every figure must have a figure number and a caption.

I'm repeating things now to make this longer so as to test formatting.  You, on the other hand, are not allowed to repeat things just to make your paper longer!!

It is also possible/likely that you will have figures in this or other
sections.  \textbf{Figures must be your own or you
must have permission from the journal/author).}  Because provenance
cannot be determined from Wikipedia or other popular websites, use of
figures from these sites is prohibited.   

\section*{Gnuplot Figures}
If you are using gnuplot to generate plots of data, you may find it useful to generate \textit{eps} output from gnuplot.   If you do so, you may need to use \LaTeX commands, which generate a \textit{DVI} file, rather than PDF\LaTeX commands that go straight to a \textit{PDF} file format.   Gnuplot makes very clean \textit{eps} files which are small in size and superb resolution.   However, these do not generally play nicely with PDF\LaTeX.    Once you have a \textit{DVI} file there's (see the tools menu for whatever \LaTeX front end GUI you're using) a \textit{DVI} $\rightarrow$ \textit{PDF} converter.
\section*{Budget}
This project includes a budget proposal of around ---, which covers some specialty items including a digital camera, microspheres, and a fluid to increase the viscosity of the solution. 



If you anticipate there will be costs associated with this project, give an approximate budget here.  Projects should not exceed \$500 unless there are extenuating circumstances (that have been okay'ed by the committee).

Your mentor should help you with estimated costs for doing your project.

\section*{Timeline}

Capstone Prep and Capstone I papers {\bf MUST} have a Timeline section with an explicit
plan for the project.  The following example is just a suggestion.

\vspace{0.2in}
{\bf Capstone I}
\begin{itemize}\itemsep1pt \parskip0pt 
\item Write preliminary code (2 weeks)
\item Debug computer code (3 weeks)
\item Take data with experiment (4 weeks)
\item Write new computer program (4 weeks)
\item Analyze preliminary data, write paper, talk (2 weeks)
\end{itemize}
{\bf Capstone II} 
\begin{itemize}\itemsep1pt \parskip0pt
\item Begin migration to new structure for all computer code (3 weeks)
\item Rebuild experiment to do what it was supposed to do all along (4 weeks)
\item Rewrite new computer program (3 weeks)
\item Analyze data (3 weeks)
\item Write paper, talk (2 weeks)
\end{itemize}

\begin{acknowledgments}
Thank many people.
\end{acknowledgments}
%
%change the name of the bibliography file to your own name
%make sure you have h-physrev5.bst file in the same directory as your tex file and bib file
%then compile using Latex-Bibtex-Latex-Latex sequence!
%

\bibliographystyle{h-physrev5.bst}
% now the actual bibliography file.   Note that it does not need the .bib extension in this line!
\bibliography{YournameBibliographyFile}
\end{document}