\documentclass[aps,prb,twocolumn,groupedaddress,nofootinbib,floatfix]{revtex4}
%
\usepackage{graphicx} 
\usepackage{graphics}
\usepackage{epsfig}
\newcommand{\beq}{\begin{equation}}
\newcommand{\eeq}{\end{equation}}
%\setlength{\skip\footins}{0.3in}
\begin{document}
%
\title{Improving Expiremental Efficacy of Studying Brownian Motion}

%
\author{Blake cole}
%
% DON'T CHANGE ANYTHING IN THE NEXT FEW LINES OR DELETE BLANK LINES
%
\affiliation{This work was submitted as part of a course requirement for completion of the BS degree in the Physics Program at RIT and, in its current form, does not appear in any publication external to RIT.}
%
% PUT YOUR ADVISOR NAME BELOW.  DON'T DELETE ANY LINES
%
\altaffiliation [Rochester Institute of Technology, School of Physics and Astronomy, Faculty Advisor: ]{Dr. Louis McLane}

\date{\today}

\begin{abstract} \noindent 

The efficacy of Brownian Motion expirements has improved significantly since the advent of digital recording devices. In the past, this phenomenon was observed by viewing suspended particles through a microscope and camera lucida, and tracking the particles by marking paper at regular intervals. Utilizing high frame rate digital cameras and specialized tracking software improves the efficacy of this expirement significantly. So much so, that this method can now be used to study the visco-elastic properties of the solution the particles are suspended in. This project aims to create an expiremental process for studying Brownian Motion in an undergraduate laboratory setting, as well as potentially performing microrheology expirements.  

\end{abstract}

\maketitle

\section*{Background}
Brownian Motion was first observed by Robert Brown when he saw chaotic motion of tiny pollen particles suspended in water, originally he thought this might be due to biological organisms. Albert Einstein later theorized that this motion resulted entirely from the atomic nature of matter and could be explained as collisions between the particle and the molecules in the solution. Einstein's hypothesis was tested by the French physicist, Jean Baptiste Perrin, who had developed a technique for creating precise spherical particles around $1 \times 10^{-6}$ meters (1$\mu$m) in size. To observe the chaotic motion, the particles were suspended in water and observed through a microscope and camera lucida, which allowed the image of the microscope slide to be superimposed on a drawing surface. The position of the particles were then marked at regular time intervals. This experimental process can be greatly improved by using a digital camera and specialized particle tracking software. 
\section*{Theory}
The motion of the particle can be modeled as the sum of the impulses during a given instance ($dt$) on the particle. Consider the interaction between one molecule of the solution and the particle being studied. During the collision the particle is subject to a force $F_i(t)$. Over a given time $dt$ the impulse of this force is defined as:
\beq
J_i=\int F_i(t)dt
\label{eq:1}
\eeq
Given the relative size difference between the particle and the individual molecules which comprise the solution, one can imagine many thousands of impulses in a short time $dt$. 
The total force on the particle during a given time $dt$ is the sum of all such impulses.
\beq
F(t)dt=\sum_{i}J
\label{eq:2}
\eeq
Where $F(t)$ is given by Paul Langevin's equation describing a viscous fluid(REFER):
\beq
F(t)=-\alpha{}v(t)dt+F^{(r)}(t)dt
\label{eq:Langevin Force}
\eeq
Where $\alpha$ is the viscous drag coefficient, studied by Stokes and found to be
\beq
\alpha{}=3\pi{}\eta{}d
\label{eq:drag coeff}
\eeq
for sphere of diameter $d$ in a fluid with viscosity given by $\eta$ (REF BM4). That term causes the particle's velocity to tend towards zero, while the $F^{(r)}(t)$ represents the random force applied by thousands of particulate collisions, causing it to accelerate once again and giving rise to the particle's chaotic motion. The randomly applied force vector $F^{(r)}(t)$ creates a distribution in the particle's velocity. Specifically, one that satisfies the equipartition theorem, which states that the kinetic energy of the molecules in a system is proportional to the temperature.  In one dimension this can be expressed as:
\beq
\frac{1}{2}m\langle{}v_{x}^{2}\rangle=\frac{1}{2}K_{B}T
\label{eq:equipartition}
\eeq
where $m$ is the mass, $T$ is the temperature, $K_B$ is the Boltzmann constant, and $\langle{}v_{x}^{2}\rangle$ is the average velocity for a large population size.
The underlying statistics governing the chaotic motion come from an application of the central limit theorem, which states that if many numbers are randomly drawn from the same probability distribution, the sum of these numbers will be a Gaussian-distributed random number (ref).

Specifically the theorem predicts that $N$ random numbers drawn from any probability distribution, with a mean of $\mu_i$ and variance $\sigma_{i}^2$, then the sum of those numbers $\sum N$ will result in a Gaussian distribution with mean $\mu = N\mu_i$ and a variance of $\sigma^2=N\sigma_{i}^2$. 

In the study of Brownian Motion the mean displacement ($\Delta{}r$) of the particle is equal to zero, since the displacement in either direction is equally likely. However the mean squared displacement (MSD) is the variance of the resulting Gaussian. In one dimension:
\beq
\sigma^2=\langle{}\Delta{}r^2\rangle{}=\langle{}\Delta{}x^2\rangle{}=2Dt
\label{eq:Displacement}
\eeq
where $D$ is the diffusion coefficient defined as:
\beq
D=\frac{K_BT}{\alpha}
\label{eq:Diffusion Coeff}
\eeq
Einstein was the first to theorize a correlation between the Diffusion Coefficient ($D$) and the variance of the distribution, creating an important link between viscosity, temperature of the solution, and the random motion exhibited by a suspended particle. (CITE BM 11)



It is recommended that you start with an Introduction or Background
section.  When you cite things \cite{Foner1} put the citations in a separate file and
use bibtex.   It is good to have both historical and modern references. \cite{Tickle,Basso}
Sometimes you need to cite things other than articles. \cite{UnitsTable}
%
\section*{Data Collection and Analysis}
These predictions can be reproduced utilizing a digital camera with a fast frame rate and a microscope with a sufficient zoom. In general the procedure requires the digital camera to be sufficiently mounted to the microscope lens in such a way that the particles can be recorded reliably. Next a microscope slide containing a sample of microsphere particles is prepared so that wind currents and other outside interference is minimized. The data is collected automatically using a specialized particle tracking software. The resulting displacement data can be fit to a Gaussian and the variance calculated in order to extract the diffusion coefficient according to equations \ref{eq:Variance} and (7). Additionaly the two-dimensional MSD can be calculated at each time, relative to the camera frame rate, using the following equation:
\beq
\langle\Delta{}r^{2}(t)\rangle{}=N_{i}^{-1}\sum_{i=1}^{N_i}\left[\left(x\left(t_{i}+t\right)-x\left(t_i\right)\right)^2+\left(y\left(t_{i}+t\right)-y\left(t_i\right)\right)^2\right]
\label{eq:two2dMSD}
\eeq
A collective ensemble of the mean squared displacements is averaged and the slope of the line used to calculate the diffusion coefficient. A number of interesting values can be calculated from the diffusion coefficient including; the Bolzmann constant, Avagrado's number, the radius of the bead, or the temperature of the solution (REFER Catopovic 486).

\section*{Project Goals}	
The main goal of this capstone project is to create and document an experimental procedure, so that the experiment can be performed by undergraduates in a physics laboratory class. As such, a complete documentation for this expiremental procedure will be created for reproducibility and use in the lab course. Portions of this project will be dedicated to error analysis and potential pitfalls that may be encountered, in order to create a reliable and repeatable experiment. Additionally, if the expirement proves reliable, a similar technique could be used to study the visco-elastic properties of unknown solutions, in a process known as microrheology. 

\section*{Budget}
This project includes a budget proposal of around ---, which covers some specialty items including a digital camera and the microspheres. The particle tracking software will likely utilize a library creating by Eric Weeks and others, specifically created for Brownian Motion expirements.
Those key components and other items needed to conduct the expirement are listed below:


 \begin{tabular}{|l|r|}
\hline
  Bausch and Lomb Microscope & Lent from department\\ \hline

  Digital Camera DMK 33UX178 & 40 \\ \hline
  
  Microsphere Solution & 200  \\ \hline
 
  Diluting/Viscous Agent & 30  \\ \hline
  \hline
   Boiling Water & 100 \\ \hline
  \hline
   Boiling Silicon Oil & 160 \\ \hline
  \hline
\end{tabular}

  


\section*{Timeline}
This project will focus on creating a Brownian Motion laboratory expirement to be performed in a physics laboratory course. Ideally this will be completed by the end of Capstone I and Capstone II will be spent developed a technique for microrheology expirements. 

\vspace{0.2in}
{\bf Capstone I}
\begin{itemize}\itemsep1pt \parskip0pt 
\item Collect and Assemble Equipment (2 weeks)
\item Expirementation (5 weeks)
\item Perform Error Analysis and Tweaks (3 weeks)
\item Finalize Procedure and Document (2 weeks)
\item Analyze preliminary data, write paper, talk (2 weeks)
\end{itemize}
{\bf Capstone II} 
\begin{itemize}\itemsep1pt \parskip0pt
\item Additional Expirementation in Microrheology (3 weeks)
\item Collect Data(4 weeks)
\item Document Microrheology Procedure(3 weeks)
\item Write paper, talk (2 weeks)
\end{itemize}

\begin{acknowledgments}
Thank many people.
\end{acknowledgments}
%
%change the name of the bibliography file to your own name
%make sure you have h-physrev5.bst file in the same directory as your tex file and bib file
%then compile using Latex-Bibtex-Latex-Latex sequence!
%

\bibliographystyle{h-physrev5.bst}
% now the actual bibliography file.   Note that it does not need the .bib extension in this line!
\bibliography{YournameBibliographyFile}
\end{document}